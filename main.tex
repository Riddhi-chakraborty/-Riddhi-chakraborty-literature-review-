\documentclass[12pt,a4paper]{article}

% Packages
\usepackage[utf8]{inputenc}
\usepackage[margin=1in]{geometry}
\usepackage{hyperref}
\usepackage{enumitem}
\usepackage[style=apa,backend=biber]{biblatex} % or ieee, numeric, etc.

% Bibliography file
\usepackage[style=apa,backend=biber]{biblatex} % You can use ieee, numeric, etc.
\addbibresource{reference.bib}

\title{Literature Review Notes}
\author{Your Name}
\date{\today}

\begin{document}

\maketitle
\tableofcontents
\newpage

\section{Introduction}
Brief description of your literature review topic.

\section{Notes from Literature}

\subsection{Example Source}
\textbf{Reference:} \cite{Samir2025} \\
\textbf{DOI/URL:} \url{https://doi.org/10.1038/s41598-025-89277-6} \\
\textbf{Source type:} Journal article \\
\textbf{Main topic/keywords:} HTPEMFC Modelling,  \\

\textbf{Excerpt/Quote:}
\begin{quote}
``High membrane water content improves proton conductivity, but excess water in the gas channels can cause flooding.'' (p. 45)
\end{quote}

\textbf{Your notes/interpretation:} This is important for my humidifier modeling. I can reference this when discussing optimal inlet RH.  

\textbf{Tags:} water management, humidification

\newpage
\section{All References}
\printbibliography

\end{document}